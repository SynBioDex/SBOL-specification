\section{Mapping Between SBOL 1, SBOL 2, and SBOL3}
\label{sec:mapping}

In broad strokes, the SBOL 1 standard focused on conveying physical, structural information, whereas SBOL 2 expanded the scope to include functional aspects as well.  
The physical information about a designed genetic construct includes the order of its constituents and their descriptions. 
Specifying the exact locations of these constituents and their sequences allows genetic constructs to be defined unambiguously and reused in other designs. 
SBOL 2 extended SBOL 1 in several ways: it extends physical descriptions to include entities beyond DNA sequences, and it added support for functional descriptions of designs.  
SBOL 3 refined the data model to simplify the representation of common use cases.

\subsection{Mapping between SBOL 1 and SBOL 2}

\ref{SBOL1TO2} depicts the mapping of SBOL 1.1 classes to SBOL 2.x classes, indicating corresponding classes/properties by color.
The SBOL 2.x \sbol{Model} and \sbol{CompenentDefinition} classes have no SBOL 1.1 equivalent, and thus are not shown.
In particular:
\begin{itemize}
\item SBOL 1.1 \external{Collection} objects containing \external{DnaComponent} objects map to SBOL 2.x \sbol{Collection} objects that contain \sbol{CompenentDefinition} objects with DNA \sbolmult{types:CD}{types} properties.
\item SBOL 1.1 \external{DnaComponent} objects maps to SBOL 2.x \sbol{CompenentDefinition} objects with DNA \sbolmult{types:CD}{types} properties.
\item SBOL 1.1 \external{DnaSequence} objects maps to an SBOL 2.x \sbol{Sequence} objects with \external{IUPAC DNA} \sbol{encoding} properties.
\item SBOL 1.1 \external{SequenceAnnotation} objects with \external{bioStart} and \external{bioEnd} properties map to SBOL 2.x\\
\sbol{SequenceAnnotation} objects that contain \sbol{Range} objects.
\item SBOL 1.1 \external{SequenceAnnotation} objects that lack \external{bioStart} and \external{bioEnd} properties map to an SBOL 2.x \sbol{SequenceFeature} objects that contain \sbol{GenericLocation} objects.
\item Each SBOL 1.1 \external{SequenceAnnotation} also maps to an SBOL 2.x \sbol{Component}, which represents the instantiation or usage of the appropriate \sbol{CompenentDefinition}.
\item Each SBOL 1.1 \external{precedes} property maps to an SBOL 2.x \sbol{SequenceConstraint} that specifies a precedes \sbol{restriction} property.
\end{itemize}

\begin{figure*}[h]
\begin{center}
  \includegraphics[width=\textwidth]{images/sbol_v1_to_v2}
\end{center}
\caption{\label{SBOL1TO2}The mapping from the SBOL 1.1 data model to the SBOL 2.x  data model, indicating corresponding classes/properties by color.}
\end{figure*}

\subsection{Mapping between SBOL 2 and SBOL 3}

\begin{itemize}
    \item SBOL 2.x \external{ComponentDefinition} objects map to SBOL 3.x \sbol{Component} objects.  The \sbol{type} property is mapped according to  \autoref{tbl:component_type_mapping}.
    \item SBOL 2.x \external{ModuleDefinition} objects map to SBOL 3.x \sbol{Component} objects with a \sbol{type} of \texttt{SBO:0000241} (functional entity)
    \item SBOL 2.x \external{Component}, \external{Module}, and \external{FunctionalComponent} objects map to SBOL 3.x \sbol{SubComponent} objects
    \item SBOL 2.x \external{SequenceConstraint} objects map to SBOL 3.x \sbol{Constraint} objects
    \item SBOL 2.x \external{MapsTo} objects with a value other than \external{merge} are converted to an SBOL 3.x \sbol{Constraint}. \todo{Jake: please can you provide more detail?}
    \item SBOL 2.x \external{SequenceAnnotation} objects 
\end{itemize}




There is a bidirectional mapping from an Interface to a set of "access" and "direction" properties. It will only lose information for FunctionalComponents that have inputs and outputs that are "private", which should not generally happen (i.e., if it's an input or output, it should be part of the public interface).

From SBOL 2 to SBOL 3:

For converting from a ModuleDefinition, every FunctionalComponent with a "direction" property that is not "none" is listed in the Interface object (if there are no such, don't create an Interface). The mapping from direction to interface properties is: "in"-->"inputs", "out"-->"outputs", "inout" --> "nondirectional". Finally, every Component with "access"="public" and "direction"="none" is listed as "nondirectional" in the Interface.

For converting from a ComponentDefinition, every Component with "access"="public" is listed as "nondirectional" in the Interface.







\begin{table}[ht]
  \begin{edtable}{tabular}{ll}
    \toprule
    \textbf{Component Type} & \textbf{URI for SBO Term} \\
    \midrule
      \url{http://www.biopax.org/release/biopax-level3.owl\#Dna} & \url{https://identifiers.org/SBO:0000251 (DNA)}\\
      \url{http://www.biopax.org/release/biopax-level3.owl\#DnaRegion} & \url{https://identifiers.org/SBO:0000251} (DNA)\\
      \url{http://www.biopax.org/release/biopax-level3.owl\#Rna} & \url{https://identifiers.org/SBO:0000250} (RNA)\\
      \url{http://www.biopax.org/release/biopax-level3.owl\#RnaRegion} & \url{https://identifiers.org/SBO:0000250} (RNA)\\
      \url{http://www.biopax.org/release/biopax-level3.owl\#Protein} & \url{https://identifiers.org/SBO:0000252} (Protein)\\
      \url{http://www.biopax.org/release/biopax-level3.owl\#SmallMolecule} & \url{https://identifiers.org/SBO:0000247} (Simple Chemical)\\
      \url{http://www.biopax.org/release/biopax-level3.owl\#Complex} & \url{https://identifiers.org/SBO:0000253} (Non-covalent Complex)\\
    \bottomrule
  \end{edtable}
  \caption{Mapping of SBOL2 ComponentDefinition types to SBOL3 Component types}
 \label{tbl:component_type_mapping}
\end{table}:
