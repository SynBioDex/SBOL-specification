% -----------------------------------------------------------------------------
\section{Best Practices}
\label{sec:bestpractices}
% -----------------------------------------------------------------------------
\subsection{Indicate Modification of Data with Version}

Once an SBOL object has been published where others might have accessed it (e.g., to an online repository), it may be the case that others make copies of the object or else come to depend on the particular contents of the object.
Thus, in order to avoid confusion, if a person wants to change the properties of a published object, the best practice is to do so by making a new copy that incorporates the change, with a new URI.

The relationship between the old and new objects (i.e., that the new object was derived from the old object), however, is not visible unless it is explicitly declared.
This is RECOMMENDED to be done using the \sbol{persistentIdentity}, and \sbol{version} properties.
The preferred practice for declaring such a relationship is to use the same \sbol{persistentIdentity} for both objects, and label the newer one as being the newer version.
Then, when the new object is published, it can be clear to both humans and machines that this object is intended to replace the one that was published previously.

As stated in \ref{sec:version},  it is RECOMMENDED that version numbering should follow the conventions of semantic versioning (\url{http://semver.org/}), particularly as implemented by Maven (\url{http://maven.apache.org/}).
This convention represents versions as sequences of numbers and qualifiers separated by the characters {\tt .} and {\tt -} and compared in lexicographical order (for example, 1 < 1.3.1 < 2.0-beta).
For a full explanation, see the linked resources.

\Rtodo{try to target readers unfamiliar with RDF/XML.  -bder}
\Rtodo{maybe clarify what type of versioning is this object affecting? TN}

\subsection{Creation and Modification Dates}

\LDtodo{Annotations: Annotating with created and modified dates, and how to add them.}

\subsection{Compliant SBOL Objects}
\label{sec:compliant}

Maintaining unique identity URIs for all SBOL objects is a very challenging implementation task.  To reduce the developer's burden, users of SBOL 2.0 are encouraged to follow a few simple rules when constructing the identity and related fields for SBOL objects.  When these rules are followed, we say that the SBOL object is \emph{compliant}.  The rules are as follows:
\begin{enumerate}
\item The \sbol{identity} of an SBOL object should begin with a \emph{URI prefix} that maps to a domain over which the user has control.  Namely, the user can guarantee uniqueness of identities within this domain.
\item In a compliant SBOL object, the \sbol{persistentIdentity} and \sbol{displayId} properties are required.
\item When a SBOL object is not given a \sbol{version}, the \sbol{identity} and \sbol{persistentIdentity} must be equal.
\item When a SBOL object is given a \sbol{version}, the \sbol{identity} must be equal to the "\refObj{persistentIdentity}/\refObj{version}".
\item The \sbol{persistentIdentity} of a compliant \sbol{Collection} object must end with "/col/\refObj{displayId}". 
\item The \sbol{persistentIdentity} of a compliant \sbol{ModuleDefinition} object must end with "/md/\refObj{displayId}". 
\item The \sbol{persistentIdentity} of a compliant \sbol{Model} object  must end with "/mod/\refObj{displayId}".
\item The \sbol{persistentIdentity} of a compliant \sbol{ComponentDefinition} object must end with "/cd/\refObj{displayId}". 
\item The \sbol{persistentIdentity} of a compliant \sbol{Sequence} object must end with "/seq/\refObj{displayId}". 
\item The \sbol{persistentIdentity} of a compliant \sbol{GenericTopLevel} object must end with "/gen/\refObj{displayId}". 
\item The \sbol{persistentIdentity} of a compliant child object must begin with the \sbol{persistentIdentity} of its parent object and be immediately followed by "/\refObj{displayId}". 
\item The \sbol{version} of a compliant child object must be equal to the \sbol{version} of it parent object.
\end{enumerate}

\subsection{Annotations: Embedded Objects vs. External References}

\LDtodo{Don't drag your giant data files around in SBOL, put them as external links}

\Ctodo{Would be good to also talk about completeness checking here.  Namely, if all objects are in the file, should verify that links are all valid.  When they are not, need a scheme to validate external references.}

\subsection{Recommended Ontologies for External Terms}
External ontologies and controlled vocabularies are integral part of SBOL. SBOL utilises these resources to access existing biological information where possible. New SBOL specific terms are defined only when necessary. Instead, SBOL provides placeholders that can point to external terms. For example, types of components, such as DNA or protein, are indicated using BioPAX. Similarly, the role of a DNA component is indicated via the SO terms. Although preferred ontologies have been indicated in relevant sections where possible, other resources providing similar terms can also be used. A summary of these external sources can be found at \ref{tbl:preferred_external_resources}.

\Ctodo{It would be useful to add URLs to the resources - NeilW}


\begin{table}[ht]
  \begin{edtable}{tabular}{p{3cm}p{3cm}p{4cm}}
    \toprule
    \textbf{SBOL Entity} & \textbf{Property} & \textbf{Preferred External Resource}\\
    \midrule
    \textbf{ComponentDefinition}  & type & BioPAX \\
    						   	  & role & SO (if type is \textit{DNA} or \textit{RNA})    \\
    						   	  & role & CHEBI (if type is \textit{small molecule})    \\
    						   	  & role & UniProt (if type is \textit{protein}??) \\   
    \textbf{Interaction}	      & type & SBO      \\
    \textbf{Participation}	      & type & SBO      \\
    \textbf{Model}	      		  & language & EDAM      \\
    				      		  & framework & SBO      \\
    \bottomrule
  \end{edtable}
  \caption{SBOL properties and preferred external resources to choose values from.}
  \label{tbl:preferred_external_resources}
\end{table}

\Ctodo{Goksel and Neil need to sort out GO vs. UniProt, and possibly just recommend both here.}

\Ctodo{Somebody who cares about this should decide whether any of gets put on the JavaDocs, but it is getting deleted from this document

Within an implementing Object-Oriented (OO) API, SBOL properties should be mapped to member accessors that are similarly named and that return idiomatic representations of these properties. For example, a Java implementation would use common Java idioms. In this case, the member accessor for an optional SBOL property could return a Java primitive value, Java object, or null, while the accessor for a multi-valued SBOL property could return a Java \external{Collection}. In general, OO member accessors for multi-valued SBOL properties should never return null.

As another example, a relational implementation of the SBOL API would store the properties and associations a mixture of data fields and references via foreign keys. The fields in individual tables will correspond to the `arrowhead' end of an association (in reverse to the direction in the RDF and OO representations), and the name may be modified to reflect this change in directionality. For example, the \sbol{sequence} association between a \sbol{ComponentDefinition} and \sbol{Sequence} would be represented by a foreign key field on the \sbol{Sequence} table that references a row in the \sbol{ComponentDefinition} table, and it may be named \external{sequenceOf}.}