% -----------------------------------------------------------------------------
\section{Purpose}
% -----------------------------------------------------------------------------
% Synthetic biology aims to apply engineering principles such as standardization, modularity, and design abstraction to molecular biology.  However, synthetic biology still faces substantial challenges, including long development times, high rates of failure, and poor reproducibility. 

% The Synthetic Biology Open Language is intended to help synthetic biologists collaborate by allowing them to exchange designs in a standardized data format.  In addition, the SBOL data model systematically describes the essential details of a design that are required for researchers to reproduce each other's designs in the laboratory.  The purpose of the Synthetic Biology Open Language is to aid collaboration between researchers, improve scientific reproducibility, and to speed the research and development of technologies based on synthetic biology.

% Below is my revision. Any thoughts? - Nic

Synthetic biology builds upon the techniques and successes of genetics, molecular biology and metabolic engineering by applying engineering principles to the design of biological systems. These principles include standardization, modularity, and design abstraction. The field still faces substantial challenges, including long development times, high rates of failure, and poor reproducibility.
\todo[inline]{Maybe discuss more about how SBOL is a standard. Huge transition from what is Synethic Biology to SBOL}

To help address these challenges, the Synthetic Biology Open Language (SBOL) introduces  a standardized file format for the electronic exchange of biological designs and a standardized data model for the reproducible description of essential design details. Ultimately, SBOL is intended to speed up the research and development of designed biological systems by enhancing the exchange and reproducibility of biological designs between different labs.   
\todo[inline]{it's not about a file format, it's about a representation; the fact that this is a file format is incidental -JSB}


\todo[inline]{Motivate/discuss URI's?}


\todo[inline]{Should the comparison with 1.1 really be there?  This needs revision to be stand-alone, and right now it's basically an extract of the SBOL2 paper -JSB}
Version 1.1 of the SBOL standard focused on representing the structural aspects of genetic designs. To serve as an effective medium for the computational exchange of genetic designs, SBOL must be extended to capture more aspects of a designed system, including both structural and functional information, and the composition of complex structural and functional designs by combining simpler parts. The SBOL data model proposed in this specification provides for addressing the most pressing needs for expanding SBOL Version 1.1. 

\begin{enumerate}

\item represent structural components of a biological design, including DNA, RNA, proteins, small molecules and other physical components

\item describe behavioral aspects of a biological design, the intended or expected interactions and dynamic behavior

\item associate structure and function together, so that a single design can be understood both in terms of its structure and its behavior

\item support rich annotations of all components, so that data required to describe a design, but not formalized in this specification can be safely exchanged

\end{enumerate}

Taken together, these capabilities allow SBOL sufficient expressivity to support the description and exchange of hierarchical, modular representations of both the intended structure and function of designed biological systems.

To address the need for functional descriptions in SBOL, the proposed data model adds classes for modules, interactions, and models. These classes provide a firm basis for functional representation in SBOL without going so far as to create a new standard for mathematically modeling biology, as there already exist several established languages for doing so, from the Systems Biology Markup Language (SBML)~\cite{SBML} to CellML~\cite{CellML} and even MatLab~\cite{matlab}. Rather, these classes enable users of SBOL to group components that function together, describe the basic qualitative interactions between these components, and document references to standard mathematical models that are external to SBOL and that provide more detailed descriptions of component function. In other words, a module gathers together a set of component instantiations, a set of interactions between these component instantiations, and a set of references to external models that are expected to be consistent with the module's interactions.

\todo[inline]{This just sort of stops.  Need a better end. --JSB}