\section{SBOL Compliance}

\twozeroone{
There are different types of software compliance to the SBOL specification.  First, a software tool can either support all classes or only the structural subset of SBOL 2.0.  The structural subset includes the following classes:
\begin{itemize}
\item \sbol{Sequence}
\item \sbol{ComponentDefinition}
\begin{itemize}
\item \sbol{Component}
\item \sbol{SequenceAnnotation}
\item \sbol{SequenceConstraint}
\end{itemize}
\item \sbol{Collection}
\item \sbol{Annotation}
\item \sbol{GenericTopLevel}
\end{itemize}
Second, an SBOL compliant software tool can support import, export, or both.  
Finally, if it supports both import and export, it can either do so in a lossy or lossless fashion.

In order to test import compliance, developers are encouraged to utilize SBOL test files found here:\\ {\url{https://github.com/SynBioDex/SBOLTestSuite}}\\
Examples with every meaningful subset of objects are provided including both structural-only SBOL (i.e., annotated DNA sequence data) and complete tests.  

In order to test export compliance, developers are encouraged to validate their generated SBOL files with the SBOL Validator found here:\\
\url{http://www.async.ece.utah.edu/sbol-validator/}\\
This validator can also be used to check lossless import/export support, since it can compare the data content of a file between a files imported and exported into a software tool.

Finally, developers of SBOL-compliant tools are encouraged to notify the SBOL editors (editors@sbolstandard.org) when they have determined that their tool is SBOL compliant, so their tool can be advertised on the SBOL website.
}

