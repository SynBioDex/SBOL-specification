% % -----------------------------------------------------------------------------
\section{Overview of SBOL}
% % -----------------------------------------------------------------------------

\Rtodo{This section needs review by someone not named John or Chris}

Synthetic biology designs can be described using:
\begin{itemize}
\item Structural terms, e.g., a set of annotated sequences or information about its chemical makeup.
\item Functional terms, e.g., the way that components might interact with each other and the overall behavior.
\end{itemize}
In broad strokes, SBOL 1.1 focuses on physical, structural information, whereas SBOL 2.0 includes functional aspects. The physical information about a designed genetic circuit includes the order of its constituents and their descriptions. The exact locations of these constituents and their sequences allow genetic circuits to be defined unambiguously, and reused in other designs. SBOL 2.0 extends SBOL 1.1 in several ways: it extends physical descriptions to include entities beyond DNA sequences, and it allows for functional descriptions of the design. 

As an example, consider the design of an expression cassette, such as the one found in the plasmid pUC18~\cite{L08752.1}. This device that is designed to detect successful versus unsuccessful molecular cloning. 
As an overall system, the device is designed to grow either blue-colored (unsuccessful) or white-colored (successful) colonies in the presence of IPTG and the chemical X-gal. Internally, the device has a number of parts, including a promoter, the lac repressor binding site, and the lacZ coding sequence. 
These parts have specific component-level interactions with IPTG and X-gal, as well as native host gene products, transcriptional machinery and translational machinery that collectively cause the desired system-level behavior. 

Knowledge of how such a device functions within the context of a host and how it might be adapted to new experimental applications is currently passed on through working with fellow scientists or reading articles in papers and books. 
But there is no systematic way to communicate the integration of sequences with functional designs, so users typically have to look in many different places to develop an understanding of this system.  
The SBOL standard allows designers to describe these functional characteristics and connect them to the physical parts and sequences that make up the design. 

SBOL includes two main classes that match the structural/functional distinction above:
\begin{itemize}
\item The \sbol{ComponentDefinition} object describes the physical aspects of the designed system, such as the DNA or RNA sequences and the physical relationships among sub-components.
\item The \sbol{ModuleDefinition} object describes the local interactions of the designed system, such as specific binding relationships and repression and activation relationships. 
\end{itemize}

Figure 1 shows a simplified view of these classes, as well as other helper classes in SBOL. To continue with the pUC18 example, the description would begin with a top-level \sbol{ModuleDefinition}.  
The \sbol{ModuleDefinition} specifies the structural elements that make up the cassette by referencing a number of \sbol{ComponentDefinition} objects. These would include the DNA component for the promoter and the small molecule component for IPTG, for example.  
The \sbol{ComponentDefinition} objects can be organized hierarchically.  For example, the plasmid \sbol{ComponentDefinition} may reference \sbol{ComponentDefinition}s for the promoter, coding sequence, etc.  
Each \sbol{ComponentDefinition} object can also include the actual \sbol{Sequence} information (if available), as well as \sbol{SequenceAnnotation} objects that identify the locations of the promoters, coding sequences, etc., on the \sbol{Sequence}.  
In order to specify functional information, the \sbol{ModuleDefinition} can specify \sbol{Interaction} objects that describe any qualitative relationships among components, such as how IPTG and X-gal interact with the gene products.  Finally, a \sbol{ModuleDefinition} object can point to a \sbol{Model} object that provides a reference to a complete quantitative model using a language such as SBML, CellML, Matlab, etc.  Finally, all the of elements of the genetic design can be grouped together within a \sbol{Collection}.

\begin{figure}[ht]
\begin{center}
\includegraphics[scale=0.7]{images/OverviewFigforSpec-v7.png}
\caption{Main classes of information represented by the SBOL standard, and their relationships.  Red boxes are classes from the SBOL 1.1 that focused on structure, whereas blue classes are some of the new classes that support the functional aspects of designs.}
\label{images:overview1}
\end{center}
\end{figure}

Whereas Figure~\ref{images:overview1} provides a broad overview of SBOL, Figure~\ref{images:overview2} provides a detailed, implementation-level overview of the class structure for the SBOL 2.0 data model. This figure relies on the semantics of the \emph{Unified Modeling Language} (UML), which will be presented in more detail in the next section. Figure 2 distinguishes between \emph{top level} classes, in green, and other supporting classes (note that Figure 1 also includes all of the top level classes). In Figure 2, dashed arcs represent "refersTo", whereas a solid arrow represents ownership. In UML, the meaning of ownership is that if a parent class is deleted, so are all of its owned children. Thus, a \sbol{Collection} does not own its
\sbol{ComponentDefinition} objects, because these can stand on their own. All of the supporting classes (in orange) must be owned by some top-level class, directly or indirectly. 

% Figure~\ref{images:overview2} provides a more detailed view the the class structure for the SBOL 2.0 data model.  The main, or \emph{top level} classes, are \sbol{Collection}, \sbol{ComponentDefinition}, sbol{Sequence}, \sbol{ModuleDefinition}, and \sbol{Model}.  The key distinction of these classes is that they can stand alone and be referenced by other top level objects (see the dashed arrows between the green boxes).  The purpose of these classes is described above.  Each of these classes is assisted in their purpose by several \emph{child} classes.  The key distinction of a child object is that it is owned by its parent object, and if that parent object is removed, so is the child object.  This ownership is indicated using the solid arrows in the figure.  For example, a \sbol{ComponentDefinition} owns its \sbol{SequenceAnnotation}s.  

\begin{figure}[ht]
\begin{center}
\includegraphics[scale=0.85]{images/OverviewFig2-v4.png}
\caption{Main classes of information represented by the SBOL 2.0 standard, and their relationships.  Green boxes are ``top level'' classes, while the other classes are in support of these classes. Solid arrows indicates ownership, whereas a dashed arrow indicates that one class refers to an object of another class.}
\label{images:overview2}
\end{center}
\end{figure}

Another important difference between the figures is to more appropriately connect the functional side (modules) to the physical side (components). This is accomplished via the class \sbol{FunctionalComponent}. This class allows modules to own their components instances, and yet also allows the physical descriptions (in \sbol{ComponentDefinition}s) to stand on their own. In a similar manner, the ability to have hierarchies of either functional or physical components shown in Figure~\ref{images:overview1} must be broken apart, so that sub-components can be used in multiple functional modules or multiple physical components. Thus, instead of the arc from \sbol{ModuleDefinition} to itself as in Figure~\ref{images:overview1}, our implementation actually divides this notion into two classes, \sbol{ModuleDefinition} and \sbol{Module}. Therefore, a \sbol{ModuleDefinition} does not own the \sbol{ModuleDefinition}s that it uses, but instead it refers to them using the \sbol{Module} objects that it does own.  The identical relationship occurs on the physical side with \sbol{ComponentDefinition} and \sbol{Component}. Finally, SBOL 2.0 provides a few other additional helper classes such as \sbol{Location} that generalizes the positioning information from SBOL 1.1 to allow discontinuous ranges and cuts to be annotated, and \sbol{SequenceConstraint} that generalizes the relative positioning information among \sbol{Component}s.  There is also 
\sbol{Participation}s, which allow \sbol {Interaction} objects to specify the roles of their participants while referencing the \sbol{FunctionalComponent}s, so that these can stand on their own. Finaly, there is the \sbol{MapsTo} class (not shown) that enables connections to be made between \sbol{Component}s and \sbol{FunctionalComponent}s at various levels of the design hierarchy.  The next section provides complete definitions and details for all of these classes.

There is one final critical part of SBOL 2.0, its extension mechanism.  This extension mechanism enables both a framework for application specific information, and a means to prototype representation of data whose format has not yet reached community consensus.  In particular, each SBOL entity can be annotated using the \emph{Resource Description Framework} (RDF). Moreover, application specific entities in the form of RDF documents can be included as \texttt{GenericTopLevel} entities. SBOL libraries makes these annotations and entities available to tools as generic properties and objects that are preserved during subsequent read and write operations.