% % -----------------------------------------------------------------------------
\section{Overview of SBOL}
% % -----------------------------------------------------------------------------

Synthetic biology designs can be described using:
\begin{itemize}
\item Structural terms, e.g., a set of annotated sequences or information about the chemical makeup of components.
\item Functional terms, e.g., the way that components might interact with each other. 
\end{itemize}
In broad strokes, the SBOL 1 standard focused on conveying physical, structural information, whereas SBOL 2 expanded the scope to include functional aspects as well.  The physical information about a designed genetic construct includes the order of its constituents and their descriptions. Specifying the exact locations of these constituents and their sequences allows genetic constructs to be defined unambiguously and reused in other designs. SBOL 2 extended SBOL 1 in several ways: it extends physical descriptions to include entities beyond DNA sequences, and it added support for functional descriptions of designs.  SBOL 3 refines the data model to simplify the representation of common use cases.

As an example, consider the design of an expression cassette, such as the one found in the plasmid pUC18~\cite{L08752.1}. This device is designed to detect successful versus unsuccessful molecular cloning.  As an overall system, the device is designed to grow either blue-colored (unsuccessful) or white-colored (successful) colonies in the presence of IPTG and the chemical X-gal. Internally, the device has a number of parts, including a promoter, the lac repressor binding site, and the lacZ coding sequence. 
These parts have specific component-level interactions with IPTG and X-gal, as well as native host gene products, transcriptional machinery and translational machinery that collectively cause the desired system-level behavior. 

Knowledge of how such a device functions within the context of a host and how it might be adapted to new experimental applications has generally been passed on through working with fellow scientists or reading articles in papers and books. 
But there has been no systematic way to communicate the integration of sequences with functional designs, so users typically have had to look in many different places to develop an understanding of a system.  The SBOL 2 standard enabled designers to describe these functional characteristics and connect them to the physical parts and sequences that make up the design via \sbol{ComponentDefinition}s for structural aspects and \sbol{ModuleDefinition}s for functional aspects of the design.
In SBOL 3, these two classes are merged into a single object called \sbol{Component} which describes both structural and functional aspects as depicted in Figure~\ref{images:overview1}.  Namely, to represent structural aspects, a \sbol{Component} can include \sbol{Feature}s that refer to \sbol{Location}s within a \sbol{Sequence}.  A \sbol{Component} can also include \sbol{Constraint}s between these features.  To represent functional aspects, a \sbol{Component} can include \sbol{Interaction}s that can refer to relationships between participating \sbol{Features}.  Finally, a \sbol{Component} can have its behavior described using a \sbol{Model}.

\begin{figure}[ht]
\begin{center}
  \includegraphics[scale=0.85]{images/SBOL3-main-classes.pdf}
\caption{The SBOL \sbol{Component} object and related objects.  Solid arrows indicates ownership, whereas a dashed arrow indicates that one class refers to an object of another class.  Red boxes represent structural objects, while blue boxes represent functional objects.  To represent structural aspects, a \sbol{Component} can include \sbol{Feature}s that refer to \sbol{Location}s within a \sbol{Sequence}.  A \sbol{Component} can also include \sbol{Constraint}s between these features.  To represent functional aspects, a \sbol{Component} can include \sbol{Interaction}s that can refer to relationships between participating \sbol{Features}.  Finally, a \sbol{Component} can have its behavior described using a \sbol{Model}.}
\label{images:overview1}
\end{center}
\end{figure}

To continue with the pUC18 example, the description would begin with a top-level \sbol{Component}.  The top-level \sbol{Component} specifies the structural elements that make up the cassette by referencing a number of \sbol{SubComponent} objects. These would include the DNA \sbol{SubComponent} for the promoter and the simple chemical
\sbol{SubComponent} for IPTG, for example.  The \sbol{Component} objects can be organized hierarchically.  For example, the plasmid \sbol{Component} might reference \sbol{SubComponent}s for the promoter, coding sequence, etc.  Each \sbol{Component} object can also include the actual \sbol{Sequence} information (if available) via an entire \sbol{Location} \sbol{SequenceFeature}, as well as \sbol{SubComponent} objects that identify the \sbol{Location}s of the promoters, coding sequences, etc., on the \sbol{Sequence}.  In order to specify functional information, the \sbol{Component} can also specify \sbol{Interaction} objects that describe any qualitative relationships among \sbol{SubComponent} \sbol{Participation}s, such as how IPTG and X-gal interact with the gene products.  Finally, a \sbol{Component} object can point to a \sbol{Model} object that provides a reference to a complete computational model using a language such as SBML~\cite{SBML}, CellML~\cite{CellML}, MATLAB~\cite{matlab}, etc.

Whereas \ref{images:overview1} provides a broad overview of SBOL, \ref{images:overview2} provides a detailed overview of the main classes within the SBOL 3 data model.  In particular, designs can be expressed using \sbol{CombinatorialDerivation}s, \sbol{Component}s, and \sbol{Sequence}s.  A \sbol{CombinatorialDerivation} allows one to specify a design pattern where individual \sbol{SubComponent}s can be selected from a set of variants.  The \sbol{Implementation} class is the build class, and it is used to represent physical artifacts like a plasmid.  The \sbol{Experiment} and \sbol{ExperimentalData} classes are the test classes.  They allow the description of an experiment and the corresponding data generated by that experiment.  The \sbol{Activity} class is taken from the provenance ontology (Prov-O), which is described in Section~\ref{sec:provenance}.  For example, a build \sbol{Activity} describes how an \sbol{Implementation} is constructed using a \sbol{Component} description.  On the other hand, a test \sbol{Activity} describes how an \sbol{Experiment} is conducted using an \sbol{Implementation} artifact.  The \sbol{Collection} class has members, which can be of any of these types or \sbol{Collection}s themselves.  All of these objects can refer to objects of the \sbol{Attachment} class, which are used to link out to external data (images, spreadsheets, textual documents, etc.). 
The next sections provide complete definitions and details for all of these classes.

\begin{figure}[ht]
\begin{center}
\includegraphics[scale=0.85]{images/SBOL3-top-levels.pdf}
\caption{Main classes of information represented by the SBOL 3 standard, and their relationships.  Blue boxes represent design classes, green boxes represent build classes, yellow boxes represent test classes, red boxes represent learn classes, and the gray boxes represent additional utility classes.  Each of these classes will be described in more detail below.}
\label{images:overview2}
\end{center}
\end{figure}
