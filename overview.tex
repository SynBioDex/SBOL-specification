% % -----------------------------------------------------------------------------
\section{Overview of SBOL}
% % -----------------------------------------------------------------------------
Typically, information about a  genetic circuit includes the order of its constituents and their descriptions. The exact locations of these constituents and their sequences allow genetic circuits to be defined unambiguously, and reused in other designs. Interactions between these constituents are then used to construct biologically plausible designs. 
\todo[inline]{reword for better clarity and coverage of the model -JSB}

% In the figure below, a simple toggle switch system is displayed, in which LacI and TetR repress each other's genes transcriptionally. The toggling of the system  is controlled by adding IPTG to deactivate LacI, and ATC to deactivate TetR. The components of the system includes genetic elements, proteins, small molecules.

\begin{figure}[ht]
\begin{center}
\includegraphics[scale=1.2]{images/SBOL2_2_revised.png}
\caption[]{ }
\label{images:overview}
\end{center}
\end{figure}
\todo[inline]{Put a caption on the figure -JSB}
\todo[inline]{Remove SBML from the figure -JSB}

The \sbol{Sequence} is a fundamental information object for synthetic biology and is needed to reuse components, to replicate synthetic biology work, and to assemble new synthetic biological systems. 
Therefore, both experimental work and theoretical sequence composition research depend heavily on the sequence associated with component definitions.
\todo[inline]{need to make it clear that it includes DNA, RNA, and protein, also smooth the text --JSB}


SBOL includes different entities to describe such genetic circuits. Genetic elements such as promoters, RBS, CDSs and terminators are defined with the \sbol{ComponentDefinition} entity. Their instances are reused in different designs via the \sbol{Component}s that refer to corresponding \sbol{ComponentDefinition}s. \sbol{ComponentDefinition}s can also represent proteins, RNAs or small molecules. They are associated with sequence information such as nucleotides aminoacids or chemical structure. A full description of a genetic circuit is then represented using  \sbol{ModuleDefinition}s which contains information about molecular interactions and their participating components. Modules can be associated with quantitative or qualitative models using the \sbol{Model} entity, which is used to point to the actual location of a model.

\todo[inline]{Need to also explain annotation --JSB}

SBOL facilitates the design of complex systems using hierarchical composition. In addition to using simple genetic elements in a modular fashion, modules that are composed of multiple, different components can also be reused. Such modules can expose some of the design components as inputs and outputs, which can be connected to components from other modules using \sbol{MapsTo} entities.
\todo[inline]{This needs to be clarified.  Do we really want to explain MapsTo here? -JSB}

% \todo[inline]{I would suggest a more complete and more explicit mapping of vocabulary to the toggle switch example. -bder}

% The same toggle switch is now displayed using two LacI and TetR inverter submodules in figure \ref{images:toggleswitch_modular}. The LacI inverter uses LacI as input and produces the TetR output, and the TetR inverter uses TetR as input and produces the LacI output. These inputs and outputs are mapped in a parent module.

